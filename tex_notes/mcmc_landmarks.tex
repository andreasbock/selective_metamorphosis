\documentclass{article}

% --- Packages --- %
\usepackage{amsmath,amssymb}
\usepackage[margin=2cm]{geometry}

\usepackage{algorithm}
\usepackage[noend]{algpseudocode}

% --- Commands --- %
% Comments
\newcommand{\cjcsays}[1]{{\bfseries Colin says:} \emph{#1}}
\newcommand{\absays}[1]{{\bfseries Andreas says:} \emph{#1}}
\newcommand{\aasays}[1]{{\bfseries Alexis says:} \emph{#1}}

% Maths
\newtheorem{definition}{Definition}
\newtheorem{theorem}{Theorem}
\newtheorem{remark}{Remark}
\newtheorem{proposition}{Proposition}

\newcommand{\pp}[2]{\frac{\partial #1}{\partial #2}}
\newcommand{\dede}[2]{\frac{\delta #1}{\delta #2}}
\newcommand{\dd}[2]{\frac{\diff#1}{\diff#2}}

\newcommand{\norm}[2]{\| #1 \|_{ #2 }}
\newcommand{\bnorm}[1]{\norm{ #1 }{B}}
\newcommand{\ltwonorm}[1]{\norm{ #1 }{2}}
\newcommand{\diff}[1]{\text{d} #1}

\newcommand{\Q}{\mathbf{q}}
\newcommand{\U}{\mathbf{u}}
\newcommand{\Z}{\mathbf{z}}
\newcommand{\V}{\mathbf{v}}

\begin{document}
\title{Landmark Metamorphosis with MCMC}
\author{Andreas, Alexis and Colin}
\maketitle

\section{Metamorphosis with Inference}

\begin{enumerate}
  \item in the formulation below, $z$ is exactly the momentum $p_i$ of landmarks, thus we don't need to do any inference, we can solve it directly. 
  \item Instead, we would want a Lagrangian of the form
    \begin{align}
      l(q,p) = \|u\|_B^2 + \sum_i |p_i|^2 \nu(q_i) \, , 
    \end{align}
    where $\nu(x)(>0)$ is a function on the landmark $q$ space ($\mathbb R^2$), with values in $\mathbb R$. 
  \item The baysian inference problem is then for this function, which we don't know. 
  \item if $\nu=0$, we recover standard landmark dynamics, if $\nu(x)=\sigma^2$ is a constant, we recover the classic landmark metamorphosis.  
  \item the equations of motion should be close to 
    \begin{align}
      \dot p_i  &= \nabla u(q_i)p_i  + |p_i|^2 \nabla \nu(q_i)\\
      \dot q &= u(q_i) + \nu(q_i) p_i 
    \end{align}
  \item another choice is 
    \begin{align}
      l(q,p) = \|u\|_B^2 + \sum_i p_i\cdot \nu(q_i) \, , 
    \end{align}
    where $\nu:\mathbb R^2 \to \mathbb R^2$, which will give more freedom on the metamorphosis term, and give equations of motion as 
\begin{align}
      \dot p_i  &= \nabla u(q_i)p_i  + p_i \cdot \nabla \nu(q_i)\\
      \dot q &= u(q_i) + \nu(q_i) 
    \end{align}
\end{enumerate}

Preliminary numerical experiments with only one dimension looks promising (2D on
its way).\\

\textbf{What we need to complete:}

\begin{enumerate}
\item Testing functionals in 2d and dump the results (sanity check)
\item Existence and uniqueness of minimisers
\item Existence and uniqueness of the ODEs
\item Well-posedness of the Bayesian formulation
\item Experiments with the tests dumping: Chains (centroid evolution), traceplots, 10 running batch MAP averages
\item Conclusion
\end{enumerate}
\section{Bayesian Interpretation}

Let $i=0,..M$, $M>0$ denote the landmark index, and let $d$ denote the spatial
dimension. The landmark metamorphosis optimisation problem reads:

\begin{subequations}\label{pbl:landmark_mm}
\begin{align}
\min_{\substack{u\in L^1([0,1],B)\\ \{z^i_t\}_i \in L^1([0,1],\mathbb{R}^{d\times M})}} &
\sum_{i=0}^M \int_0^1 \bnorm{u^i_t}^2 + \ltwonorm{z^i_t}^2 \diff{t}\\
            & \dot{q^i_t} = u_t \circ q^i_t + z^i_t \label{eq:ele_q} \\
            & q^i_j = \hat q^i_j, \quad j=0,1 \label{eq:ele_q_bcs}
\end{align}
\end{subequations}

where the boundary conditions $\hat q^i_j$ denotes the $i$th landmarks for time
$j=0,\,1$. Here $(B,\bnorm{\cdot})$ is some appropriate Hilbert space for the
velocity.

The Euler-Lagrange equation for $z$ can be shown to be
\begin{equation}\label{eq:ele_z}
\dot{z^i_t} = - (\nabla u_t \circ q^i_t)^T z^i_t
\end{equation}

We aim to solve the inverse problem \eqref{pbl:landmark_mm} using a Bayesian
formulation cf. \cite{dashti2017bayesian,dashti2013map}. This is in contrast to
traditional methods in image registration, where metamorphoses are compures as
solutions to a Riemannian least-squares problem via gradient method.\\

The goal of this approach is to compute a probability distribution on the space
of velocities and sources $(u,z)$ given some obversations $I_0$ and $I_1$. As
such, the desiderata here is a density. Mathematically this is expressed by
Bayes' rule:
\[
\dd{\mu[u,z]}{\mu_0} \propto p(I_0, I_1 | [u, z])
\]
here $\mu_0$ denotes our prior density and $\mu[u, z] \triangleq N(0,C)\times
N(0,G_u)$ for appropriate covariances $C$ and $G_u$ (details to follow). At this
instance it is useful to relate minimisers of \eqref{pbl:landmark_mm} to the
analogous notion in the stochastic setting, namely the \textit{maximum a
posteriori estimator}.

%%%% Finite dimensional MAP %%%%
We motivate this approach using an example where everything is finite-dimensional:
In finite dimension, if we sample $u$ (prior) from a Gaussian distribution,
then its PDF is proportional to $e^{-u^TC^{-1}u/2}$ for some covariance matrix
$C$. Recall that for metamorphosis, the reconstruction system in Eulerian frame
is:
\begin{subequations}\label{eq:reconst}
\begin{align}
& \partial_t q + u\cdot\nabla q = z\\
& \partial_t z + \text{div}(u z) = 0
\end{align}
\end{subequations}
Then we \textbf{define} a likelihood function $\Phi_c(u)=\frac c2\ltwonorm{z}^2$
via the solution of the reconstruction system, leading to a posterior pdf
proportional to $e^{-u^TC^{-1}u/2 - \Phi_c(u)}$.
Given
\begin{align*}
u_{MAP} & \triangleq \arg\max_u p(z|u)p(u)\\
        & \triangleq \arg\max_u \log [ p(z|u)p(u)]\\
        & \triangleq \arg\max_u \log e^{-u^TC^{-1}u/2 - \Phi_c(u)}\\
        & \triangleq \arg\max_u - u^TC^{-1}u/2 - \Phi_c(u)
\end{align*}
Since $\max_u - u^TC^{-1}u/2 - \Phi_c(u)= \min_u u^TC^{-1}u/2 + \Phi_c(u)$ we
see that the MAP estimator is also a minimiser of the metamorphosis functional.
The infinite-dimensional case is underway.\\

%%%% Explanation of why we want a joint distribution %%%%
Note that we do not want to put a stochastic model on the velocity without the
source since this indicates that we can never sample realisations corresponding
to new growth patterns! This can be seen by asking the question: ``\textit{how
unique is the equation $\partial_t I + u\cdot\nabla I = z$}''? This equation is
overdetermined, so the same image may appear for different velocities and
sources (yet to be shown). Therefore, if we choose the proposal density
$\exp(-\Phi_c(u))$ we sample only the sources that minimise the source $z$ and
new features can never appear.

\section{Numerical Solution}

There are several ways of solving \eqref{pbl:landmark_mm}. We can throw Newton
at the ELEs, solve an adjoint optimisation problem or do MCMC. This section
touches on these approaches and their implementation. We discretise the velocity
space $V$ by expanding and truncating it in a finite kernel basis
$\{K^i\}_{i=0}^M$, where $K^i : \mathbb{R}^d \rightarrow \mathbb{R}^d$:

\begin{equation}
u_t(x) = \sum_{i=0}^M K^i_t(x - q^i_t)\hat u^i_t \label{eq:ele_u_disc}
\end{equation}

where $\{\hat u^i_t\}_i \in \mathbb{R}^{d\times M}, t\in[0,1]$ denotes the
velocity basis coefficients. Suppressing the time-dependencies for now we
observe that the velocity norm can be written:
\begin{align*}
\bnorm{u}^2 & \triangleq \int_\Omega u\cdot Bu \diff{x}\\
            & = \int_\Omega\sum_{i=0}^M K(x-q^i)\hat u^i \cdot B \sum_{j=0}^M
            K(x-q^j)\hat u^j \diff{x}\\
            & = \int_\Omega \sum_{i=0}^M K(x-q^i)\hat u^i \cdot \sum_{j=0}^M
            \delta (x-q^j)\hat u^j \diff{x}\\
            & = \sum_{i=0}^M \sum_{j=0}^M K(q^j-q^i)\hat u^i \cdot\hat u^j
\end{align*}

Substituting \eqref{eq:ele_u_disc} into
\eqref{pbl:landmark_mm} we write the spatially discretised problem
\begin{subequations}\label{pbl:landmark_mm_disc}
\begin{align}
\min_{\substack{\{\hat u^i_t\}_i \in \mathbb{R}^{d\times M},\, t\in[0,1]\\
\{z^i_t\}_i \in \mathbb{R}^{d\times M},\, t\in[0,1]}} & \int_0^1 \sum_{i=0}^M
\sum_{j=0}^M K(q^j-q^i)\hat u^i \cdot\hat u^j+ \sum_{i=0}^M
\ltwonorm{z^i_t}^2 \diff{t}\\
            & \dot{q^i_t} = \sum_{j=0}^M K^j_t(q^i_t-q^j_t)\hat u^j_t \circ
            q^i_t + z^i_t \label{eq:ele_q_disc} \\
            & \eqref{eq:ele_q_bcs}\nonumber
\end{align}
\end{subequations}

Optimising \eqref{pbl:landmark_mm_disc} is somewhat awkward due to the presence
of $q$ in the functional. We can perform a change of variables to eliminate it.
First we rewrite the problem in matrix notation: let $\Q = [q_0, q_1,
\ldots]\in\mathbb{R}^{d\times M}$ denote the vector of landmark points and
similarly $\U = [u_0, u_1, \ldots]\in\mathbb{R}^{d\times M}$ and $\Z = [z_0,
z_1, \ldots]\in\mathbb{R}^{d\times M}$ the velocities and metamorphosis sources
at each of the landmarks. Further, denote the kernel matrix $M$ by the following
identity:
\[
\U^T M \U \triangleq \sum_{i=0}^M \sum_{j=0}^M K(q^j-q^i)\hat u^i \cdot\hat u^j
\]
where we have suppressed the $q$-dependency on $M$. By symmetry and
positive-definiteness of the kernel there exists $M^{1/2}$ such that $M =
(M^{1/2})^T M^{1/2}$. Therefore, by introducing the variable $\V \triangleq
M^{1/2}\U$ we observe
\[
\bnorm{u} = \langle\U^T, M \U\rangle = \langle\V^T,\V\rangle
\]
and so \eqref{pbl:landmark_mm_disc} can be written as:

\begin{subequations}\label{pbl:landmark_mm_disc_matrix}
\begin{align}
\min_{\U,\, \Z} & \int_0^1 \ltwonorm{\V}^2 + \sum_{i=0}^M \ltwonorm{\Z_i}^2 \diff{t}\\
            & \dot{\Q} = M^{1/2} \V + \Z\label{eq:Q_eq}\\
            & \eqref{eq:ele_q_bcs}\nonumber
\end{align}
\end{subequations}

Note that in the infinite-dimensional space-time setting this change of
variables is not required as the metric is not parameterised by a kernel -
rather it is an outer metric that we can evaluate in the entire ambient space.\\

\subsection{Deterministic Metamorphosis}

There are two options.\\

First, we can solve the reconstruction system directly. Via the kernel basis
parameterisation of the velocity field as implicit functions of the landmark
positions we are able to pose and solve the governing equations (\eqref{eq:Q_eq}
and the discrete analogue of \eqref{eq:ele_z}) as a non-linear boundary value
system. The non-linearity will stem from the non-linearity of the operator $K$.
We express this system in Firedrake/UFL and hit the system with Newton. We might
see multiple solutions in this case.\\

Second, we can simply write \eqref{pbl:landmark_mm_disc} in pyadjoint and optimise
over $z$.\\

In both settings time is implicitly discretised with Runge-Kutta.

\subsection{MCMC Approach for Landmarks}

We define a joint distribution $\mu[u,z]$ on the space of velocities and sources
(what we call \textit{metamorphic paths}). However, since the two functions are
linked by the reconstruction equation and hence $z \in G_u$, it is not
completely obvious how to sample from $\mu$.

The aim is to derive a quasi MCMC scheme to generate approximate metamorphic
paths and therefore
approximate minimisers to the metamorphosis problem. We (sort of) know how to
sample diffeomorphisms; choose a sufficiently smooth vector space for $u$ and
supply a suitable covariance operator. However, sampling from source terms, $z$,
is not as simple because of the coupling between $z$ and $u$ through the
equation $\partial_t q + u\cdot\nabla q = z$ (and the boundary conditions).  We
can observe, however, that the governing equation for $z$ in \eqref{eq:reconst}
is well-posed for any non-zero right-hand side, modulo some integrability
conditions. The algorithm goes as follows: First we propose random velocities
based the covariance operator $B$ (something like a tri-Helmholtz operator) and
accept them based on the inner problem (i.e.  solve \eqref{eq:reconst} and use
the density $e^{-\Phi(u)}$ to accept). This essentially amounts to sampling from
the marginal distribution with respect to $u$ (i.e. $z$ is marginalised out).
Then, after a number of samples of the velocity, we begin sampling from the
marginal distribution with respect to $z$. If we look only at the equation for
the image, then this is a rather difficult problem because not every $z$ will
with probability 1 make the image match the boundary conditions! However, we
can introduce a right-hand side variable $S$ to the equation for $z$ i.e.
$\partial_t z + \text{div}(u z) = S$. If we need only something like $L^2$ in
order to control $z$ with $S$, then we can generate $\delta$-correlated samples
of $S$ through which we can generate possible metamorphic sources $z$ from the
marginalised distribution for $z$!\\

\newcommand{\mhsample}{\textsc{sampleL2}}
\newcommand{\inverthh}{\textsc{invertTriHelmHoltz}}
\newcommand{\acceptprob}{\textsc{acceptanceProbability}}
\newcommand{\solveinner}{\textsc{solveInnerProblem}}
\begin{algorithm}
\begin{algorithmic}
\caption{Quasi-MCMC on $\mu[u,z]$}
\Procedure{quasiMCMC}{$N$}
\State $k \gets 0$
\State $v^k \gets \text{ some initial guess}$
\State $S^k \gets \text{ some initial guess}$
\State $z \gets 0$
\While{$k < N$}
\State $v \gets \mhsample ()$\hspace{3cm} \textbf{\# sample velocity}
\State $S \gets \mhsample ()$\hspace{3cm} \textbf{\# sample mutation}
\State $u \gets \inverthh (v)$
\State $z_\text{prop} \gets \solveinner (u, S)$
\If {\textsc{randomUnit()}$\,< \acceptprob(z_\text{prop}, z)$}
    \State $v^{k+1} \gets v$
    \State $S^{k+1} \gets S$
    \State $z \gets z_\text{prop}$
\Else
    \State $v^{k+1} \gets v^k$
    \State $S^{k+1} \gets S^k$
\EndIf
\State $k\gets k+1$
\EndWhile
\Return $v^k,\, S^k$
\EndProcedure
\end{algorithmic}
\end{algorithm}

where:
\begin{itemize}
\item \mhsample$(f)$ returns a $L^2$ function sample computes as
$\sqrt{1-\beta^2}f + \beta \xi$, where $\xi$ is a $N(0,1)$ piece-wise linear
function
\item \inverthh inverts a tri-Helmholtz operator onto its argument and returns
the result
\item $\solveinner(u,S)$ computes the image $I$ and source $z$
given a fixed velocity $u$ and mutation $S$. The mutation defaults to zero.
\item $\acceptprob(z_u, z_v)$ return the value $e^{-c(\ltwonorm{z_u}^2
-\ltwonorm{z_v}^2)}$
\end{itemize}

Since the velocities have a tensor-product structure sampling is not obvious.
However, we can avoid Bochner-type draws because we simple focus on sampling the
$v$ component and then invert Helmholtz operators. As a result, we only need
$H^1$ covariance (or $H^2$, depending on the dimension).

\begin{remark}[Evaluating the quality of samples]
When we accept sample velocities we need to
assess whether or not they are actually appropriate potential minimisers, or
realisations, of metamorphoses. We can either keep a running tally, i.e. at
every step when we accept, we evaluate the metamorphosis functional and decide
the ``quality'' of this realisation (sometimes we want low $z$ norm, sometimes
not if we have diffeomorphic images!). This is called the running tally
approach and depends on the type of application or images that we are looking
at!
\end{remark}

See \cite{cotter2010approximation} for more information on results pertaining
the connection between approximation of forward problems of PDEs and
interpretation of Bayesian inverse problems.

\bibliographystyle{abbrv}
\bibliography{landmarks}

\end{document}
