\documentclass{article}
\usepackage{amsmath,amssymb}
\usepackage[margin=2cm]{geometry}

% Comments
\newcommand{\cjcsays}[1]{{\bfseries Colin says:} \emph{#1}}
\newcommand{\absays}[1]{{\bfseries Andreas says:} \emph{#1}}
\newcommand{\aasays}[1]{{\bfseries Alexis says:} \emph{#1}}

% Maths
\newtheorem{definition}{Definition}
\newtheorem{theorem}{Theorem}

\newcommand{\norm}[2]{\| #1 \|_{ #2 }}
\newcommand{\knorm}[1]{\norm{ #1 }{K}}
\newcommand{\ltwonorm}[1]{\norm{ #1 }{2}}
\newcommand{\diff}[1]{\text{d} #1}

\begin{document}
\title{Landmark Metamorphosis with MCMC}
\author{Andreas, Alexis and Colin}
\maketitle

\section{Bayesian Interpretation}

Following the Bayesian interpretation of inverse problems detailed in
\cite{dashti2017bayesian} we aim to compute metamorphoses in a different manner
to a Riemannian least-squares solution via a gradient method. We motivate this
approach using an example where everything is finite-dimensional:\\
%\begin{example}[General strategy to sample from stochastic metamorphoses]
In finite dimension, if we sample $u$ (prior) from a Gaussian distribution,
then its PDF is proportional to $e^{-u^TS^{-1}u/2}$ for some covariance matrix
$S$. Recall that for metamorphosis, the reconstruction system in Eulerian frame
is:
\begin{align*}
& \partial_t q + u\cdot\nabla q = z\\
& \partial_t z + \text{div}(u z) = 0
\end{align*}
Then we \textbf{define} a likelihood function $\Phi_c(u)=\frac c2\ltwonorm{z}$
via the solution of the reconstruction system, leading to a posterior pdf
proportional to $e^{-u^TS^{-1}u/2 - \Phi_c(u)}$. 
Given
\begin{align*}
u_{MAP} & \triangleq \arg\max_u p(z|u)p(u)\\
        & \triangleq \arg\max_u \log [ p(z|u)p(u)]\\
        & \triangleq \arg\max_u \log e^{-u^TS^{-1}u/2 - \Phi_c(u)}\\
        & \triangleq \arg\max_u - u^TS^{-1}u/2 - \Phi_c(u)
\end{align*}
Since $\max_u - u^TS^{-1}u/2 - \Phi_c(u)= \min_u u^TS^{-1}u/2 + \Phi_c(u)$ we
see that the MAP estimator is also a minimiser of the metamorphosis
functional.\\

In the gradient descent approach we discretise the function spaces and then
optimise the metamorphosis functional. In this interpretation we want to use
MCMC to generate samples from the conditional PDF rather than just finding the
$u$ that has maximum probability (this would solve the deterministic
metamorphosis problem) - we call this stochastic metamorphosis. MCMC allows us
to compute statistics for $u$ (and $q$ and $z$) from this distribution via
averaging over samples.
%\end{example}
\section{Optimisation Problem}

Let $i=0,..M$, $M>0$ denote the landmark index. The landmark metamorphosis 
optimisation problem reads:
\begin{subequations}
\begin{align}
\min_{u, z} & \sum_{i=0}^M \int_0^1 \knorm{u^i_t}^2 + \ltwonorm{z^i_t} \diff{t}\\
            & \dot{q^i_t} = - u_t \circ q^i_t + z^i_t \\
            & \dot{z^i_t} = - (\nabla u_t \circ q^i_t)^T z^i_t \\
            & q^i_1 = \hat q^i_1 \\
            & q^i_0 = \hat q^i_0 \\
            & u^i_t = K^i_t(q^i_t)
\end{align}
\end{subequations}

here the boundary conditions $\hat q^i_j$ denotes the $i$th landmarks for time
$j=0,\,1$. Here, $K^i_t$ denotes an operator that computes the velocity at each time
and landmark step.

\section{Numerical Solution}

If we can parameterise - at each landmark and time step - a Gaussian kernel with
a random $U(0,1)$ component then we can write the velocity as an implicit
function of the landmark positions. As a result, the governing equations
(constraints) above will become a nonlinear boundary value system. The
nonlinearity will stem from the nonlinearity of the operator $K$.

\bibliographystyle{abbrv}
\bibliography{mcmc}

\end{document}
