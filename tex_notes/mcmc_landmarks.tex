\documentclass{article}
\usepackage{amsmath,amssymb}
\usepackage[margin=2cm]{geometry}

% Comments
\newcommand{\cjcsays}[1]{{\bfseries Colin says:} \emph{#1}}
\newcommand{\absays}[1]{{\bfseries Andreas says:} \emph{#1}}
\newcommand{\aasays}[1]{{\bfseries Alexis says:} \emph{#1}}

% Maths
\newtheorem{definition}{Definition}
\newtheorem{theorem}{Theorem}

\newcommand{\norm}[2]{\| #1 \|_{ #2 }}
\newcommand{\knorm}[1]{\norm{ #1 }{K}}
\newcommand{\ltwonorm}[1]{\norm{ #1 }{2}}
\newcommand{\diff}[1]{\text{d} #1}

\begin{document}
\title{Landmark Metamorphosis with MCMC}
\author{Andreas, Alexis and Colin}
\maketitle

\section{Bayesian Interpretation}

Following the Bayesian interpretation of inverse problems detailed in
\cite{dashti2017bayesian} we aim to compute metamorphoses in a different manner
to a Riemannian least-squares solution via a gradient method. We motivate this
approach using an example where everything is finite-dimensional:\\

In finite dimension, if we sample $u$ (prior) from a Gaussian distribution,
then its PDF is proportional to $e^{-u^TS^{-1}u/2}$ for some covariance matrix
$S$. Recall that for metamorphosis, the reconstruction system in Eulerian frame
is:
\begin{align*}
& \partial_t q + u\cdot\nabla q = z\\
& \partial_t z + \text{div}(u z) = 0
\end{align*}
Then we \textbf{define} a likelihood function $\Phi_c(u)=\frac c2\ltwonorm{z}$
via the solution of the reconstruction system, leading to a posterior pdf
proportional to $e^{-u^TS^{-1}u/2 - \Phi_c(u)}$. 
Given
\begin{align*}
u_{MAP} & \triangleq \arg\max_u p(z|u)p(u)\\
        & \triangleq \arg\max_u \log [ p(z|u)p(u)]\\
        & \triangleq \arg\max_u \log e^{-u^TS^{-1}u/2 - \Phi_c(u)}\\
        & \triangleq \arg\max_u - u^TS^{-1}u/2 - \Phi_c(u)
\end{align*}
Since $\max_u - u^TS^{-1}u/2 - \Phi_c(u)= \min_u u^TS^{-1}u/2 + \Phi_c(u)$ we
see that the MAP estimator is also a minimiser of the metamorphosis
functional.\\

The infinite-dimensional case in underway. More information on MAP estimators in
connection with Bayesian inverse problems can be found in \cite{dashti2013map}.

\section{Optimisation Problem}

Let $i=0,..M$, $M>0$ denote the landmark index. The landmark metamorphosis 
optimisation problem reads:
\begin{subequations}\label{pbl:landmark_mm}
\begin{align}
\min_{u, z} & \sum_{i=0}^M \int_0^1 \knorm{u^i_t}^2 + \ltwonorm{z^i_t} \diff{t}\\
            & \dot{q^i_t} = - u_t \circ q^i_t + z^i_t \label{eq:ele_q} \\
            & q^i_j = \hat q^i_j, \quad j=0,1 \label{eq:ele_q_bcs}\\
            & u^i_t(x) = K^i_t(x - q^i_t) \label{eq:ele_u}
\end{align}
\end{subequations}

here the boundary conditions $\hat q^i_j$ denotes the $i$th landmarks for time
$j=0,\,1$. Here, $K^i_t$ denotes an operator that computes the velocity at each time
and landmark step.

The Euler-Lagrange equation for $z$ is:
\begin{equation}\label{eq:ele_z}
\dot{z^i_t} = - (\nabla u_t \circ q^i_t)^T z^i_t
\end{equation}

\section{Numerical Solution}

There are several ways of solving \eqref{pbl:landmark_mm}. We can throw Newton
at the ELEs, solve an adjoint optimisation problem or do MCMC. This section
touches on these approaches and their implementation.

\subsection{Deterministic Metamorphosis}

There are two options. We can write the equations \eqref{eq:ele_q},
\eqref{eq:ele_q_bcs}, \eqref{eq:ele_z} and \eqref{eq:ele_u} as UFL expressions
in Firedrake and hit the system with Newton. We might see multiple solutions in
this case.\\

Second, we can simply write \eqref{pbl:landmark_mm} in pyadjoint and optimise
over $z$ (substituting the expression for $u$, \eqref{eq:ele_u}, in the right
places).

\subsection{MCMC Approach}

Solving the reconstruction system should be feasible.  If we can parameterise -
at each landmark and time step - a Gaussian kernel with a random $U(0,1)$
component then we can write the velocity as an implicit function of the landmark
positions. As a result, the governing equations (constraints) above will become
a nonlinear boundary value system. The nonlinearity will stem from the
nonlinearity of the operator $K$.\\

When we accept sample velocities we need to
assess whether or not they are actually appropriate potential minimisers, or
realisations, of metamorphoses. We can either keep a running tally, i.e. at
every step when we accept, we evaluate the metamorphosis functional and decide
the ``quality'' of this realisation (sometimes we want low $z$ norm, sometimes
not if we have diffeomorphic images!). This is called the running tally
approach and depends on the type of application or images that we are looking
at!

\bibliographystyle{abbrv}
\bibliography{landmarks}

\end{document}
